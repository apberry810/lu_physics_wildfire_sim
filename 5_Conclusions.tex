pot\newpage
\section{Conclusions}\label{conclusions}

\subsection{Accuracy of the Model}

We present a novel cellular automaton approach for modelling wildfires, using an adaptation of previous work by A. Aleksandridis \textit{et al.} in 2008 \cite{Aleksandridis_2008} and A. Quartieri \textit{et al.} in 2010 \cite{Quartieri_2010}. The environmental parameters of elevation, wind and spotting are added to this model to recreate fire wildfire behaviour as accurately as possible. We analyse how the introduction of each of these parameters affects different aspects of fire behaviour, including the shape of the fire front and the rate of spread. Our model allows us to quantify these effects using a set of equations that describe the probability of ignition of an arbitrary cell, presented in Equation \ref{p2}. Despite an uncertainty of $\sim 15-20\%$ in the fuel ignition probability, our model shows good predictive potential. \newline \indent We apply our model to simulate the events of the 2009 Australian bushfires, with mixed results. Our cellular automaton model exhibits wildfire-like behaviour in the simulated physical environment of Murrindindi, with specific limitations. The fires follow the direction of the wind and travel uphill faster than downhill, and the simulated fires travel through the town of Marysville as expected. We determine that the most important physical variables in the simulation are the speed and direction of wind, when measured by the accuracy in predicting the total area burnt. These findings are supported by existing literature \cite{Xuehua_2016, Freire_2019}, giving us confidence in the validity of our approach. Although the basic behaviour of wildfire is recreated, the model requires fine-tuning before it can be accurately applied to real geographical systems, as it substantially overestimates the surface area burnt by the wildfires. To improve the simulation we propose additional factors as described in Section \ref{furtherwork}.

\subsection{Proposals for Future Work}\label{furtherwork}

To generate reliable real-world predictions from our model, we propose that the direction of the wind be implemented as a time-dependent vector map. This will allow the fire to change direction along possible wind changes, which should significantly improve the simulation accuracy. We also suggest that the values of the strength parameters $\alpha$ and $\beta$ be determined through a Monte-Carlo simulation, so that the results of the simulation match real-world events as accurately as possible. To do this, we suggest modifying the parameter optimisation procedure developed by M. Denham and K. Laneri for fire simulations in 2018 \cite{Denham_2018}. Finally, we propose that the model be tested on a variety of past fire events to verify its validity and see if the constants $\Delta t,\alpha,\beta,E,M,t_f$ and $\sigma$ are universal across different landscapes. \newline \indent A brief investigation into the wildfire effects on the local populations in Marysville was conducted alongside the main project. A summary of our methodology and initial findings is found in Appendix \ref{population_appendix}. Combining cellular automata models for both fire simulation and population movement allows an insight into how different parameters influence the eventual number of casualties in the fires. Ultimately, such a model could be used to predict high risk areas or be used to adjust bushfire survival advice before the events happen.