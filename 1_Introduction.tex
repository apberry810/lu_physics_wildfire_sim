\newpage
\section{Introduction} \label{introduction}

Wildfires are dramatic, destructive and evolve in an almost chaotic manner. They form part of the natural ecological system of our planet, but with advancing global temperatures and imbalances occurring in local environments, they may become more common place in everyday life for high risk regions \cite{JTAba_APWil}. Hence a fundamental understanding of their genesis and characteristics may be essential in limiting the damage caused. It is here we will investigate what the influential characteristics of wildfires are without taking the traditional scientific route of direct physical experiment. We will instead be taking the theoretical approach of simulation.\newline \indent The framework for our simulation is cellular automata, which has been widely studied and successfully applied in past research on living and dynamic systems \cite{CA_1,CA_2}. Cellular automata has been used to model wildfires numerous times before \cite{Aleksandridis_2008,CA-fire_2}, and we will test a novel variation of this modelling system to investigate if it can produce a wildfire simulation of specific parameters. In particular, we explore whether it can offer results akin to historical wildfire data when applied to past geographical source data.\newline \indent To obtain an understanding of the nature of the parameters we are trying to replicate, we can imagine starting a small fire at the corner of this paper. We can now probe into the way in which the fire spreads depending on a collection of factors: the type and thickness of the paper, folds and undulations in the paper, airflow across the surface of the paper and embers resulting from previously burnt regions. All of these have legitimate observable effects on the pattern and rate of burn, and combinations of them begin to give a sense of complexity which was probably not first perceived when introducing the simple concept of burning paper. \newline \indent Developing our image to a geographical scale now leads us to consider parameters such as flammability, topography and wind. We wish to know how accurately these parameters can be modelled via simulation and whether any relations can be derived that explain the attributes. Once a foundation of the principles of cellular automata and wildfire modelling are set in Sections \ref{CA_Principles} and \ref{Modelling_Wildfire} respectively, we develop the methodology behind our model in Section \ref{methodology}. We investigate the proportional influence of different parameters on the effect of wildfire spread, later finding that elevation can decrease the probability of ignition by a factor of 10 in our model with specific environments. Yet the strength and direction of wind highly coerces the behaviour of fire spread, either completely stunting the fires ability to spread or being a determining factor in where the fire spreads.\newline \indent After determining the results that the basis of our model offers in Section \ref{results_1}, we then apply it to real geographical and vegetation data from the Murrindindi Shire in Australia in Section \ref{results_2}. We quantify how accurately our model can replicate the 2009 "Black Saturday Bushfires" under the influence of various environmental parameters, and compare the results of our simulation to historical data. Ultimately it is found that wind again plays a primary role in the core behaviour of the wildfire and provides the largest step in driving the model to more accurately represent the actual events. Wind also becomes a key limiting factor in our model, as its dynamic nature is approximately fixed to a single speed and direction for the duration of the simulation, in turn restricting the true course of the wildfire.